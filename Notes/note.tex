\documentclass [12 pt, twoside] {article}
\usepackage[margin=1in]{geometry}
\usepackage[utf8]{inputenc}
\usepackage{listings}
\usepackage{color}
\usepackage{setspace}

\definecolor{codegreen}{rgb}{0,0.6,0}
\definecolor{codegray}{rgb}{0.5,0.5,0.5}
\definecolor{codeblue}{rgb}{0,0,0.6}
\definecolor{backcolor}{rgb}{0.95,0.95,0.95}

\lstdefinestyle{mystyle}{
	backgroundcolor = \color{backcolor},
	commentstyle = \color{codeblue},
	keywordstyle = \color{codegreen},
	numberstyle = \color{codegray},
	stringstyle = \color{magenta},
	basicstyle = \footnotesize,
	breakatwhitespace = false,
	breaklines = true,
	captionpos = b,
	keepspaces = true,
	numbers = left,
	numbersep = 5pt,
	showspaces = false,
	showstringspaces = false,
	showtabs = false,
	tabsize = 4
}

\lstset{style = mystyle}

\begin{document}

\title{APCS Notes}
\author{Yicheng Wang}
\date{2014-2015}

\maketitle
\newpage
\setcounter{tocdepth}{3}
\tableofcontents
\newpage

\section{2014-9-8}


\subsection{Comparison of Programming Languages}
\newline
\textbf{Scheme}:
\newline Annoying Prefix Notation
\newline strict syntax
\newline not object oriented
\newline only seperated by parenthesis
\newline IDE NOT necessary
\newline Mostly recursion + list
\newline Functional Programming Language (Everything is a function)
\newline
\newline
\textbf{Netlogo}:
\newline GUI based
\newline Shines on Interactive Modeling
\newline Bad for input/output data
\newline Parallel Programming Language
\newline Not a general-purpose language -- ONLY USEFUL IN NETLOGO ENVIROMENT
\newline Netlogo IDE (Integrated Development Enviroment) NECESSARY
\newline
\newline
\textbf{Python}:
\newline High-level Language
\newline Uses Indentation
\newline infixed math + prefix function
\newline Linear processing
\newline General-purpose language
\newline Interperted Language
\newline IDE NOT necessary
\newline
\newline
\textbf{Java}:
\newline Object oriented
\newline infixed math + prefix function
\newline Mid-level Language
\newline
\newline
$\LaTeX$
\newline Markup Language
\newline Compiled Language

\newpage
\section{2014-9-10}
\subsection{"Hello World" in Java}


Last year, we learned how to write the "Hello World" program in python.
\begin{lstlisting}[language=Python]
def hello():
	print "Hello world!"

hello()
\end{lstlisting}


Now, we'll learn about how to do this in Java:


Java is more restrictive than python, java programs are usually in their own folders.
Java's invented for portability and "amount of stupid/super-smart people."
Different smart people have different ways of approaching problems.
Java's designed to limit people's ways of doing things to make big project easy.
Real good programmers don't like java... b/c it's restrictive.
Java is designed to be industrially viable.


An object defines a specific thing within your program.
Everything in java is an object.
\newline A Class = object type.
\newline Tradition = 1 class per file, named starting with upper-case letter


Here's a simple program in Java:
\begin{lstlisting}[language=Java]
/*
	This is a null line
	C'est un comment!
*/

// C'est un end-of-line comment

import java.io.*;
import java.util.*;

public class Hello {  // public = the outside world (aka other things in your program) can see this
	public static void main(String[] args) {
		System.out.println("Hello World");
	}
}
\end{lstlisting}


\subsection{Running Java}
Source code (foo.java) $\to$ Java compiler (foo.class) $\to$ JVM
\newline Note that java doesn't compile to machine code, java compiles to javaBiteCode using JVM. This is where the portability comes in.


\subsection{Java technicalities}
method = function in python
\newline You  need a method in one of you're classes called "main"


\section{2014-9-11}
\subsection{Moving into the "java way" of doing things!}


Java is object oriented.
Object oriented means that the world is made of objects.
Every object has its unique attributes.
Objects also have abilities, aka things they can do.
Every program in java is made of objects.


Let's take the example of a simple chess program.
An example of an object would be a pawn.
It would have attributes like color and position.
Its abilities would include moving and attacking.
However, these pawns are different, White pawn 1-8 and Black pawn 1-8.
They behave in the same way, but they have different positions.
You don't want 16 seperate definitions b/c most of them are the same.
Therefore one would create a class for all of the pawns, which would define the "info about objects."
We then make objects which are known as "instances of a class."
Objects are made based on the definitions defined within the class.


Hello world program #2 -- the java way:
\begin{lstlisting}[language=Java]
// We'll use objects to do stuff

import java.io.*;
import java.util.*;

public class Greeter {
	// We put the attributes here

	// We put the abilities here
	// In Java, these are called methods
	// Methods are functions, but they belong to specific classes
	public void greet() {
	
	// public = can be called from outside the class
	// void = this doesn't send anything back, like null returner in C
 
		System.out.println("Hello world!");
	}
}
\end{lstlisting}

\section{2014-9-15}
\subsection{Typical anatomy of Java Program}


A program is consisted of objects.
One must tell java where to start the program --> public static void main
One calls that class "driver," it starts the java program.


Driver.java:
\begin{lstlisting}[language=Java]

import java.io.*;
import java.util.*;

public class Driver {
	public static void main(String[] args) {
		
		//How to use the greeter within the driver.
		
		Greeter g;
		//Creates a local variable to be of type greeter

		/*
		Variable declaration, all variables must be declared
		like global, turtles-own and patches-own variables in netlogo
		Declaration specifies the type of the variable
		local variable = a variable only visible/usable within a method, created when the method is called, destroyed when the function exits
		*/

		/*
		When main is ran, it occupies some memory on the computer
		Greeter g is a small box within main, we need to do something with it
		or java refuses to do stuff with it
		*/

		g = new Greeter();
		/*
		New:
		 1. Allocates enough memory to store a Greeter.
		 2. Do whatever's necessary to setup / initiates the memory to be a Greeter.
		 3. Returns the address of the memory that was allocated.

		 The assignment statement stores the address in g.
		*/

		System.out.println(g);
		
		// This prints the location of the variable g within the memory

		/*
		When this file is compiled, Greeter is compiled as well
		All methods/class called during main are compiled as well
		*/

		g.greet();
		/*
		Accesses the greet method within the class g.
		*/
	}
}
\end{lstlisting}
\end{document}
